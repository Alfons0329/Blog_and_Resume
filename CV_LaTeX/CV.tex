%%%%%%%%%%%%%%%%%%%%%%%%%%%%%%%%%%%%%%%%%
% Medium Length Professional CV
% LaTeX Template
% Version 2.0 (8/5/13)
%
% This template has been downloaded from:
% http://www.LaTeXTemplates.com
%
% Original author:
% Trey Hunner (http://www.treyhunner.com/)
%
% Important note:
% This template requires the resume.cls file to be in the same directory as the
% .tex file. The resume.cls file provides the resume style used for structuring the
% document.
%
%%%%%%%%%%%%%%%%%%%%%%%%%%%%%%%%%%%%%%%%%

%----------------------------------------------------------------------------------------
%	PACKAGES AND OTHER DOCUMENT CONFIGURATIONS
%----------------------------------------------------------------------------------------

\documentclass{resume} % Use the custom resume.cls style

\usepackage[left=1.5cm,top=1.0cm,right=1.5cm,bottom=1.0cm]{geometry} % Document margins 
\usepackage{hyperref}%let hyperref go
\newcommand{\tab}[1]{\hspace{.2667\textwidth}\rlap{#1}}
\newcommand{\itab}[1]{\hspace{0em}\rlap{#1}}

%----------------------------------------------------------------------------------------
%	Summary SECTION
%----------------------------------------------------------------------------------------

\begin{document}
\begin{center}
    {\huge\textbf{ALFONS HWU}}
\end{center}

{\bf \url{Mail: alfons.cs04@g2.nctu.edu.tw}} \hfill {\bf \url{Blog: https://alfons0329.github.io/afhwu0329.github.io/}}
\\
%----------------------------------------------------------------------------------------
%	Summary SECTION
%----------------------------------------------------------------------------------------
\begin{rSection}{Summary}
Major in CS NCTU Taiwan, Interested in AI, Machine Learning, Computer Security, Parallel Programming, and System-Related Topics.
\end{rSection}

%----------------------------------------------------------------------------------------
%	Education SECTION
%----------------------------------------------------------------------------------------

\begin{rSection}{Education}
{\bf BS: National Chaio Tung University, Taiwan} \hfill {\em Sep 2015 - Present} 
\\ Department of Computer Science  
\\ Overall Avg.: 87.7 / 100, Overall GPA: 3.82/4.0, Last 60: 3.92/4.0, Major: 3.88/4.0
\\ Rank: 31/194
\\ Graduation Project Advisor: \href{https://www.cs.nctu.edu.tw/members/detail/kuanwen}{Kwan-Wen Chen} 
\\
\\
{\bf MS: National Taiwan University, Taiwan} \hfill {\em Sep 2019 - }
\\ Graduate Institute of Computer Science
\end{rSection}

%----------------------------------------------------------------------------------------
%	Curricular experience and honors SECTION
%----------------------------------------------------------------------------------------

\begin{rSection}{Curricular experience and honors}

{\bf 2016 NTU Hackathon Winner} \hfill {\em Aug 2016}
\begin{itemize}
    \item Category: Smart Life, using C\# for windows and Arduino for signal detection 
    \item An eye-care sensor which notifies user to set a period for easing one's eyes along with some game-like feature to make it more interesting.
    \item \href{https://www.youtube.com/watch?v=LerD1-Vispg}{Demo and speech}
\end{itemize}

{\bf 2017 MakeNTU Hackathon Participant} \hfill {\em Feb 2017}
\begin{itemize}
    \item A hackathon mainly focus on integration between software and hardware 
\end{itemize}

{\bf Scholarship Winner} \hfill {\em Fall 2016}
\begin{itemize}
    \item Scored 98/100 in Data Structure, 2nd place throughout the whole class 
\end{itemize}

{\bf Intercolligiate Programming Contest Participant}
\begin{itemize}
    \item Participated twice in PTC, April 2017 and March 2018
\end{itemize}

{\bf Best Graduation Project Award} \hfill {\em Fall 2018}
\begin{itemize}
    \item Achieved first place over 28 teams in Dept. of Computer Science NCTU
    \item Project name: Multiplayer VR gaming assisted with Computer Vision
    \item \href{https://github.com/kai0122/NCTU-CS-Graduation-Project?fbclid=IwAR3DxQrgYiWn7BxJb9AZVgITAuv5XSOHrdM8DTjRmwvZTLSyk_Wg-JlftW0}{Slides here} 
\end{itemize}

{\bf Qualified for Exchange Student in School of Computer Science, Carnegie Mellon University}
\end{rSection}

%----------------------------------------------------------------------------------------
%	Extracurricular experience and honors SECTION
%----------------------------------------------------------------------------------------
\begin{rSection}{Extracurricular experience and honors}

{\bf MediaTek Internship, Two-Time Core Value AWARD} \hfill {\em Jul 2018 - Aug 2018}
\begin{itemize}
    \item Working on porting the Image Processing Algorithm from high-level C++ to low level embedded C, optimization and writing analytical reports
    \item Two-time VAWARD(Monthly employee performance award). Awarded by Senior Department Manager
\end{itemize}

{\bf Two-Time General Coordinator of Smart Transportation Hackathon } \hfill {\em Jul 2018 and Jan 2019}
\begin{itemize}
    \item Co-organize with my advisor, Prof. Kuan-Wen Chen aka (ITCIC 2018)
    \item Supervisory authorities: Dept of Information Technology, Ministry of Education, Taiwan
    \item \href{http://covis.cs.nctu.edu.tw/ITCIC2018/}{Hackathon webpage} \href{http://covis.cs.nctu.edu.tw/ITCIC2018/}{, webpage2}
\end{itemize}

{\bf Swimming Varsity in National Chiao Tung University} \hfill {\em Sep 2015 - Sep 2016}
\begin{itemize}
    \item Participated in two inter-collegiate swimming competitions, 4th place on 400m freestyle competition in Hsinchu Mayor Cup 2016
\end{itemize}

{\bf Piano Club Member in National Chiao Tung University} \hfill {\em Sep 2016 - Present}
\begin{itemize}
    \item Onstage performance two times.
\end{itemize}

\end{rSection}
%----------------------------------------------------------------------------------------
%	Selected Projects SECTION
%----------------------------------------------------------------------------------------
\begin{rSection}{Selected Projects}

{\bf Digital Verilog Lab FPGA: Ping Pong Battle} \hfill {\em Nov 2016 - Jan 2017}
\begin{itemize}
    \item Implement table tennis game with Verilog on XILINX FPGA, IO with Keyboard and VGA. Featuring simple battle AI, cheat mode and smash like real table tennis.
    \item \href{https://www.youtube.com/watch?v=R4cgMN5uRBE}{Demo video}
\end{itemize}

{\bf ARM STM32 Embedded Lab: AURA RGB Light } \hfill {\em Dec 2017 - Jan 2018}
\begin{itemize}
    \item Embedded-like C, and burn into ARM STM32 board to implement a dynamic, fully-customizable RGB ambient light similar to that of gaming keyboards. 
    \item Understanding the embedded ARM system programming, simple circuit, ADC, and PWM signal control.
    \item \href{https://www.youtube.com/watch?v=FdnTKmdjxIc}{Demo video}
\end{itemize}

{\bf Autopilot Drone} \hfill {\em Mar 2018 - Jun 2018}
\begin{itemize}
\item With OpenCV and C++ to implement the drone flying itself via ARUco marker + face recognition and lands on track.
\end{itemize}

{\bf Parallel Programming: Parallelize Gaussian Blur} \hfill {\em Dec 2018}
\begin{itemize}
    \item Accelerate the Gaussian Blur with CUDA, 935 times faster on GTX1070 vs i5 7500
    \item \href{https://github.com/Alfons0329/Parallel_Programming_Fall_2018/tree/master/Final\%20Project}{GitHub repo}, \href{https://github.com/Alfons0329/Parallel_Programming_Fall_2018/blob/master/Final\%20Project/Team24_Final_Project_Report.pdf}{Report}
\end{itemize}

{\bf Graduation Project: Multi-Player Cooperative VR Game} \hfill {\em Sep 2018 - Dec 2018}
\begin{itemize}
    \item Computer vision-aided multiple people real-time VR cooperative gaming using Kinect, Unity and VIVE VR
    \item \href{https://www.youtube.com/watch?v=rYiFose7gZU}{Demo video} \href{https://youtu.be/Kfz1KYMKrtU}{, Presentation video}
\end{itemize}
\end{rSection}


%----------------------------------------------------------------------------------------
%	Relevant Courses SECTION
%----------------------------------------------------------------------------------------
\begin{rSection}{Relevant Courses}
\itab{\textbf{Core Courses}} \tab{}  \tab{\textbf{Other Courses}}
\\ \itab{Data Structure \& Algorithm} \tab{}  \tab{Probabilities \& Statistics}
\\ \itab{Computer Organization} \tab{}  \tab{Signals and Systems} 
\\ \itab{Intro to Machine Learning} \tab{}  \tab{Calculus} 
\\ \itab{Compiler Design} \tab{} \tab{Linear Algebra}
\\ \itab{Operating System} \tab{} \tab{Network Security}
\\ \itab{ARM STM32 Embedded Lab} \tab{} \tab{Parallel Programming}
\\ \itab{Object-Oriented Programming} \tab{} \tab{FreeBSD System Administration}
\end{rSection}

\end{document}


